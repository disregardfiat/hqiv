\documentclass[12pt,a4paper]{article}
\usepackage[utf8]{inputenc}
\usepackage{amsmath,amssymb,amsthm}
\usepackage{mathtools}
\usepackage{physics}
\usepackage{graphicx}
\usepackage{hyperref}
\usepackage{natbib}
\usepackage{booktabs}

\title{Horizon-Quantized Informational Vacuum (HQIV): \\ A Covariant Baryon-Only Cosmological Framework from Quantised Inertia}

\author{Steven Ettinger Jr\thanks{Excelsior University (Undergraduate Student), Independent Researcher}}

\date{February 22, 2026}

\begin{document}
\sloppy

\maketitle

\begin{abstract}
We derive a covariant cosmological framework from the single axiom of conserved total informational energy with causal-horizon cutoffs on vacuum modes (Quantised Inertia). In its ultimate first-principles form the framework aims for no free parameters: the auxiliary horizon field $\phi(x) = 2c^{2}/\Theta_{\rm local}(x)$ is purely geometric (fixed by the expansion scalar of fundamental observers), the thermodynamic coefficient $\gamma$ is fixed by Brodie's (2026) Rindler-cosmic horizon overlap integral, and the matter density is envisioned as a statistical relic of early-universe horizon quantization. We present the background dynamics and linear perturbation equations derived from the full action principle. A $\beta$-free vorticity-coupled modal solver yields positive vorticity growth (exponent $+1.9$) and an older universe. Runs of our implemented CLASS fork recover the first five acoustic peak ratios $\ell_n/\ell_1$ to within $\sim$4--18\% of \citeauthor{Planck2018} \citep{Planck2018}, with a systematic shift to lower $\ell$ attributable to the modified post-recombination expansion history. These preliminary runs, together with a unified resolution of the apparent age, acoustic-peak, and $\sigma_8$ tensions via the same covariant mechanisms, motivate full Planck likelihood optimisation and the planned early-universe horizon-mode simulation.
\end{abstract}

\section{Introduction}
Standard cosmology requires unseen components for 95\% of the energy budget and faces tensions with JWST early galaxies, the Hubble tension, and large-scale coherence. A particularly attractive consequence is automatic compatibility with the Hubble tension. By normalising to the higher local $H_0 \approx 73$\,km\,s$^{-1}$\,Mpc$^{-1}$ \citep{Riess2022} while the modified Friedmann equation yields a $\sim$32--34\,Gyr age (CLASS-consistent, full radiation and precise integration), the framework produces a slower expansion history at $z \gtrsim 10$ (lower $H(z)$ at recombination) together with a modified sound horizon from reduced baryon inertia. Preliminary results from our CLASS-HQIV fork (Section~5.1) show promising agreement in relative peak ratios while exhibiting the expected shift from the modified expansion history.

We present a minimal baryon-only alternative based on horizon-modified inertia, building on the quantised inertia framework developed by McCulloch \citep{mcculloch2007, mcculloch2016} and the thermodynamic derivation of Brodie (2026). The goal is to derive every equation from first principles.

The quantised inertia framework has seen renewed theoretical interest from the horizon physics community. Notably, \citet{brodie2026} derives a MOND-like modification to inertia from Jacobson thermodynamics \citep{Jacobson1995}, finding the critical acceleration scale within 9\% of Milgrom's empirical value \citep{Milgrom1983a}. This independent theoretical derivation from thermodynamic first principles connects the QI/horizon physics programme to the broader MOND literature and strengthens the case for horizon-based modifications to inertia.

\subsection{Relation to Existing Literature and Critiques}

Quantised Inertia (QI) has attracted significant criticism from the mainstream physics community, and we address the principal concerns transparently here.

\textbf{Criticisms of the original Unruh derivation.} 
Renda \citep{Renda2019} identified two issues with the original QI derivation: the treatment of the Casimir energy term and the assumption of horizon-scale isotropy. We acknowledge these concerns. Our framework addresses the anisotropy issue by working with the covariant auxiliary field $\phi(x)$ directly, which encodes the local horizon structure without assuming spherical symmetry. The thermodynamic derivation of Brodie \citep{brodie2026} provides an independent route to the inertia modification that does not rely on the Unruh effect directly.

\subsection{Response to Renda (2019)}

Renda (2019, MNRAS 489, 881) performed a careful analysis of the original Quantised Inertia (QI) derivation as presented by McCulloch (2007, 2013) and identified two principal technical concerns:

\begin{enumerate}
    \item \textbf{Treatment of the Casimir energy term.} 
    Renda noted that the subtraction of the Casimir-like vacuum energy between a local Rindler horizon and the cosmic horizon was not derived from first principles and led to an ad-hoc modification of the Unruh spectrum.

    \item \textbf{Assumption of horizon-scale isotropy.} 
    The original formulation assumed perfectly spherical, isotropic horizons on all scales, which is unrealistic in the presence of cosmic structure, gravitational lensing, peculiar velocities, and integrated Sachs--Wolfe effects.
\end{enumerate}

We fully acknowledge both issues as valid criticisms of the \emph{early} QI literature. However, the Horizon-Quantized Informational Vacuum (HQIV) framework presented here was constructed precisely to eliminate them.

First, the Casimir-energy concern is sidestepped entirely. HQIV does not rely on the original Unruh-radiation derivation or any explicit Casimir subtraction. Instead, we adopt the fully thermodynamic route of Brodie (2026), who derives the inertia modification from Jacobson's (1995) local thermodynamic relation $\delta Q = T\,\delta S$ applied to \emph{two} horizons (local Rindler + cosmic causal) while enforcing entanglement monogamy. The microscopic origin is supplied by the single HQIV axiom of conserved total informational energy $E_{\rm tot} = mc^2 + \hbar c / \Delta x$ with $\Delta x \le \Theta_{\rm local}$. This approach never invokes a Casimir term between horizons and yields the entropy correction factor $f(a,\Theta) = a / (a + cH/6)$ directly from the overlap integral — with zero free parameters.

Second, the isotropy assumption is removed at the foundational level. Rather than assuming spherical horizons and then introducing a post-hoc smoothing parameter $\beta(t)$, HQIV works exclusively with the \emph{covariant auxiliary field}
\[
\phi(x) \equiv \frac{2c^{2}}{\Theta_{\rm local}(x)},
\]
where $\Theta_{\rm local}(x)$ is defined geometrically via the expansion scalar $\theta$ of the timelike congruence of fundamental observers (or the proper distance along the past light-cone to the nearest null surface). This field is an intrinsic part of the spacetime geometry; it automatically encodes all local anisotropies due to lensing, voids, peculiar velocities, and Sachs--Wolfe effects without any averaging or additional parameters. In the homogeneous FLRW limit it reduces exactly to $\phi = cH$, recovering the background dynamics, while spatial gradients $\nabla\phi$ generate the action-derived vorticity source that is a central prediction of the theory.

Renda also re-derived a discrete blackbody spectrum and obtained a modified form for the inertia function $F(a)$. Because HQIV never employs that specific blackbody construction, the discrepancy does not apply. The thermodynamic interpolation function $f(\alpha,\phi)$ used here is fixed by Brodie's overlap integral and matches galactic rotation curves in the low-acceleration regime while remaining fully consistent with the informational-energy axiom.

In summary, while Renda's critique correctly exposed shortcomings in the pre-2019 QI literature, the present covariant, action-based, and thermodynamically grounded formulation renders both objections obsolete. The framework satisfies the contracted Bianchi identities, conserves momentum, and is ready for quantitative testing via our implemented CLASS fork and the planned early-universe horizon-mode simulation.

\textbf{Redshift dependence of the acceleration scale.}
\citeauthor{McGaugh2016}'s analysis of rotation curves at high redshift \citep{McGaugh2016} (z $\approx$ 2.5) suggests the acceleration scale $a_0$ is approximately constant, whereas simple QI predicts $a_0(z) \propto H(z)$. In our covariant formulation, the minimum acceleration $a_{\rm min} = \chi c \phi = \chi c^2/\Theta_{\rm local}$ scales with the local horizon, which in an expanding FLRW background does follow $H(z)$. However, the full interpolation function $f(\alpha,\phi)$ and the thermodynamic floor $f_{\rm min} \approx 0.01$ may partially compensate this scaling. Our CLASS fork (Section~5.1) can be used to test whether this tension is resolved.

\textbf{Bullet Cluster constraints.}
The Bullet Cluster \citep{Clowe2006} remains a challenging empirical test for any modified-inertia theory. The observed separation between the X-ray gas (dominant baryonic mass) and the weak-lensing mass peak is difficult to reproduce without some form of collisionless dark matter. Our framework's direction-dependent inertia modification may produce this effect, but this requires explicit verification. The planned N-body simulation with ray-tracing will test whether the lensing--gas separation can be reproduced \citep{Clowe2006}.

\textbf{Conservation laws.}
Critics have noted that some formulations of QI lead to apparent violations of momentum conservation when extended to systems like the EMDrive. Our action-based formulation is fully covariant and satisfies the contracted Bianchi identities automatically, ensuring $\nabla^\mu T_{\mu\nu} = 0$. We are not proposing any violation of fundamental conservation laws.

We emphasize that this framework is speculative and that the criticisms above remain active areas of investigation. The purpose of this paper is to present an internally consistent theoretical framework and to identify the specific observational tests that can confirm or falsify it.

\section{Theoretical Framework}
The core axiom is conservation of total informational energy:
\[
E_{\rm total} = mc^2 + \frac{\hbar c}{\Delta x},
\]
with $\Delta x$ bounded by the nearest causal horizon $\Theta_{\rm local}(x)$.

The minimum acceleration $a_{\rm min} = \chi c \phi(x)$ (with $\chi \approx 0.172$ from the full light-cone average) leads to modified inertia
\[
m_i = m_g \, f(\alpha,\phi),
\]
where $f$ is the thermodynamic interpolation function defined below.

It should be noted that while the framework aims for a parameter-free formulation in its ultimate first-principles form, the current implementation contains several auxiliary coefficients that are derived from the theoretical foundations but are held fixed during numerical evolution:

\begin{itemize}
    \item $\alpha \approx 0.6$: exponent in the varying gravitational coupling, derived from the requirement that the horizon term correctly reproduces galaxy rotation curves in the low-acceleration regime.
    \item $\chi \approx 0.172$: scaling factor from the full light-cone average that relates the minimum acceleration to the horizon field.
    \item $\gamma \approx 0.35$--$0.45$: thermodynamic coefficient fixed by Brodie's Rindler-cosmic horizon overlap integral.
    \item $f_{\rm min} \approx 0.01$: thermodynamic floor set by saturation of the informational-energy budget.
    \item $\Omega_m \approx 0.06$: placeholder matter density for current numerical work, intended to emerge from early-universe horizon statistics in the full theory.
\end{itemize}

In the long term, the goal is that these values should emerge from the complete early-universe horizon-mode simulation rather than being specified externally. Until that simulation exists, we treat these as fixed theoretical inputs. The phrase ``aims for no free parameters'' in this paper should therefore be understood as ``no parameters introduced solely for observational fitting''; all coefficients have a theoretical origin.

In the original formulation, the horizon-smoothing parameter $\beta$ was introduced as a frame-dependent measure of horizon anisotropy. In a perfectly homogeneous universe, all observers would see identical spherical horizons and $\beta = 1$ exactly. However, as discussed in Section 2.3, this parameter was always intended to emerge from frame-dependent integration over anisotropic horizons rather than being freely fitted. The recent reformulation removes this parameter entirely, replacing it with the geometric field $\phi(x)$ which is fixed by the expansion scalar of fundamental observers.

\subsection[The Horizon-Smoothing Parameter beta (Historical Context)]{The Horizon-Smoothing Parameter $\beta$ (Historical Context)}

The parameter $\beta$ was originally defined as the horizon-smoothing factor:
\begin{equation}
\beta(t) = \frac{\langle \Theta_{\rm eff} \rangle}{\Theta_0} = 1 - \frac{\sigma_\Theta}{\Theta_0}
\end{equation}
where $\langle \Theta_{\rm eff} \rangle$ is the angle-averaged effective horizon distance, $\Theta_0 = 2c/H_0$ is the naive spherical horizon, and $\sigma_\Theta$ quantifies horizon anisotropy.

In practice, horizons are anisotropic due to:
\begin{itemize}
    \item Gravitational lensing by intervening structure
    \item Local voids and overdensities
    \item Doppler shifts from peculiar velocities
    \item Integrated Sachs--Wolfe effects along the past light cone
\end{itemize}

The key prediction was that as the universe ages, horizons smooth out (structures merge, peculiar velocities damp, lensing converges), so $\beta(t) \to 1$ as $t \to \infty$.

The current framework sidesteps this parameter entirely by working directly with the covariant auxiliary field $\phi(x)$, which contains the same physical information in a more fundamental form.

\subsection{Matter Content: Emergent from Horizon Statistics}

In a complete first-principles framework, the matter density should not be an input parameter at all. The baryon-to-photon ratio $\eta = n_b/n_\gamma$ and the total matter content are \textbf{statistical relics} of the early universe evolution, determined by horizon quantization during the radiation-dominated era and recombination.

The radiation density is fixed by the CMB temperature:
\begin{equation}
\rho_\gamma = \frac{\pi^2}{15} \frac{(k_B T_0)^4}{(\hbar c)^3}, \quad T_0 = 2.725\,{\rm K}
\end{equation}
This is the thermal echo of recombination — the photon bath is a fossil of when the universe became transparent.

The matter density we observe today is a \textbf{statistical outcome} of:
\begin{enumerate}
    \item Horizon-quantized mode structure during the radiation era
    \item The baryogenesis epoch (matter-antimatter asymmetry from horizon effects?)
    \item Recombination dynamics (when matter decoupled from radiation)
    \item Subsequent structure formation (matter clumping vs. horizon smoothing)
\end{enumerate}

We hypothesize that the baryon-to-photon ratio $\eta \approx 6 \times 10^{-10}$ emerges from horizon quantization at the QCD or electroweak scale:
\begin{equation}
\eta \sim \frac{N_{\rm matter\,modes}}{N_{\rm radiation\,modes}} \bigg|_{T \sim 1\,{\rm GeV}}
\end{equation}
where the mode counts are determined by what fits inside successive past light-cones.

This is a prediction, not an input. A complete simulation evolving from $T \sim 10^{15}$ GeV down to recombination should output the observed $\eta$ and $\Omega_m$ without these being specified.

Since we have not yet built the full early-universe simulation, we use the \textbf{observed} matter density as a placeholder for current numerical work:
\begin{equation}
\rho_m^{\rm observed} \approx 4 \times 10^{-28}\,{\rm kg/m^3} \quad (\Omega_m \approx 0.06)
\end{equation}

This is analogous to how standard cosmology uses observed $\Omega_b$ — but we emphasize that in our framework, this should \textbf{fall out} of the horizon statistics, not be fitted. The apparent gravitational effects attributed to dark matter arise from the horizon modification to inertia, not from missing mass.

\textbf{Key test:} Build a simulation from $z \sim 10^{10}$ to recombination. Does it predict $\eta \sim 10^{-9}$ and $\Omega_m \sim 0.05$?

\subsection{Varying Gravitational Coupling}

The effective gravitational coupling varies with horizon scale:
\begin{equation}
G(a) = G_0 \left(\frac{\Theta_0}{\Theta(a)}\right)^\alpha = G_0 \left(\frac{H(a)}{H_0}\right)^\alpha
\end{equation}
where the exponent $\alpha \approx 0.6$ is derived from the requirement that the horizon term correctly reproduces galaxy rotation curves in the low-acceleration regime \citep{mcculloch2007, mcculloch2016}. This is the same scaling that accounts for the observed $a_0 = cH_0/2$ acceleration scale in galactic dynamics.

In the covariant formulation, this varying $G_{\rm eff}(\phi)$ is renormalised by the informational cutoff (see action below).

\section{Covariant Formulation}
The modified metric is
\[
ds^2 = -(1 + 2\Phi + \phi t/c)\, c^2 dt^2 + a(t)^2 (1 - 2\Phi) \delta_{ij} dx^i dx^j,
\]
which approximates the integrated lapse correction for fundamental observers in the ADM decomposition \citep{Arnowitt2008} with $\phi$-fixed hypersurfaces (see Appendix A).

The modified Einstein equation derived from the action is
\[
G_{\mu\nu} + \gamma \left(\frac{\phi}{c^2}\right) g_{\mu\nu} = \frac{8\pi G_{\rm eff}(\phi)}{c^4} T_{\mu\nu},
\]
where $\phi(x) \equiv 2c^{2}/\Theta_{\rm local}(x)$ (units of acceleration); $\phi/c^2$ then has units of curvature (length$^{-2}$) and $\gamma$ remains dimensionless ($\approx 0.35$--$0.45$). The horizon term acts like a time-dependent effective cosmological constant that emerges from the horizon structure, not from vacuum energy.

\section{Derivation of the Full Action Principle}

We now promote the thermodynamic equilibrium condition of Brodie (2026) together with the HQIV informational axiom to a stationary-action principle. This yields a fully covariant, variational formulation that reproduces the modified inertia law, the horizon term in the Einstein equation, varying $G(a)$, and the emergent vorticity source without additional free parameters.

\subsection{Thermodynamic foundation}

Jacobson's 1995 derivation shows that the Einstein field equations emerge from the local thermodynamic relation
\begin{equation}
    \delta Q = T\,\delta S
\end{equation}
applied to a local Rindler horizon, with Unruh temperature $T = \hbar a / (2\pi k_B c)$ \citep{Unruh1976} and Bekenstein–Hawking entropy $S = A/(4\ell_P^2)$ \citep{Bekenstein1973, Hawking1975}. Brodie (2026) corrects the entropy-area law for entanglement monogamy between the local Rindler horizon (proper distance $c^2/a$) and the cosmic causal horizon ($\Theta \approx 2c/H$):
\begin{equation}
    S_{\rm eff} = f(a,\Theta)\,\frac{A}{4\ell_P^2}, \qquad
    f(a,\Theta) = \frac{a}{a + cH/6},
\end{equation}
where the factor $1/6$ arises from the backward-hemisphere overlap integral $\int_0^{\pi/2}\cos^2\theta\,\sin\theta\,d\theta$.

The HQIV single axiom supplies the microscopic origin: each vacuum mode carries a conserved total informational energy
\[
E_{\rm tot} = m c^2 + \frac{\hbar c}{\Delta x}, \qquad \Delta x \le \Theta_{\rm local},
\]
which enforces the reduced entanglement budget and thereby the entropy correction above.

Varying the total action with respect to the metric gives the modified Einstein equation
\begin{equation}
    G_{\mu\nu} + \gamma \left(\frac{\phi}{c^2}\right) g_{\mu\nu}
    = \frac{8\pi G_{\rm eff}(\phi)}{c^4}T_{\mu\nu}.
\end{equation}

We use $G_{\rm eff} = G_0$ throughout.\footnote{A higher-order Planck-suppressed correction $G_{\rm eff}(\phi) = G_0 / (1 + \gamma (\ell_P \phi/c^2)^2)$ can be included; it is negligible except near the Planck epoch ($\ell_P \phi/c^2 \sim 1$) and has no effect on post-inflationary cosmology or the CLASS runs presented here.}

Here $\Theta_{\rm local}(x)$ is the covariant causal-horizon radius measured by the fundamental observers (e.g., via the expansion scalar $\theta$ of the timelike congruence or the proper distance along the past light-cone to the nearest null surface). In a Friedmann–Lemaître–Robertson–Walker background this reduces exactly to $\Theta(t) = 2c/H(t)$, and $\phi(x)$ reduces to $\phi = cH$ (in units $c=1$).

If desired, $\phi$ can be promoted to a dynamical field by adding a simple kinetic term $\frac12(\partial\phi)^2$ and a potential $V(\phi)\propto \hbar c\,\phi^4$ whose minimum enforces $\phi\approx cH$.

\subsection{Matter sector — covariant modified inertia}

For a test particle the matter action implements the inertia modification at the world-line level:
\begin{equation}
    S_{\rm particle} = -m_g c\int f(a_{\rm loc},\phi)\,ds,
\end{equation}
where $ds=\sqrt{-g_{\mu\nu}dx^\mu dx^\nu}$ and the interpolation function (using the thermodynamic form from Brodie) is:
\begin{equation}
f(a_{\rm loc},\phi) = \max\left( \frac{a_{\rm loc}}{a_{\rm loc} + c\phi/6},\ f_{\rm min} \right),
\end{equation}
where $a_{\rm loc}$ is the local acceleration scale that sets the inertia transition (same dimensions as $c\phi/6$), distinct from the exponent $\alpha \approx 0.6$ in varying $G(a) = G_0(H/H_0)^\alpha$; $f_{\rm min} \approx 0.01$ is set by saturation of the informational-energy budget and $\chi \approx 0.172$ rescales $a_{\rm min}$ to match the empirical MOND scale.

Alternatively, the square-root form may be used:
\[
f(a_{\rm loc},\phi) = \sqrt{1 - \frac{c\phi}{a_{\rm loc}}}.
\]

Varying $S_{\rm particle}$ yields the modified geodesic equation whose non-relativistic limit is precisely
\[
m_i a^i = m_g \nabla^i\Phi, \qquad m_i = m_g f(a_{\rm loc},\phi).
\]

For a perfect-fluid continuum the Lagrangian density becomes
\[
\mathcal{L}_{\rm matter} = \sqrt{-g}\Bigl[\rho\,f(a_{\rm loc},\phi) + p + \mathcal{L}_{\rm fields}\Bigr],
\]
where the four-acceleration can be enforced via an auxiliary vector field with a Lagrange multiplier if a strictly second-order theory is required.

\subsection{Full combined action}

Collecting all pieces we obtain the final action ready for numerical implementation:
\[
S = \int\Biggl\{\frac{c^4}{16\pi G_{\rm eff}(\phi)}\,R - \frac{c^4 \gamma \phi}{8\pi G_{\rm eff}(\phi)\,c^2}
+ \rho\,f(a_{\rm loc},\phi) + p + \mathcal{L}_{\rm EM} + \mathcal{L}_{\rm other}\Biggr\}\sqrt{-g}\,d^4x.
\]
(The $c^4$ multiplies the gravitational terms; $\phi/c^2$ in the horizon term gives curvature dimensions; variation yields the Einstein equation above.)

In the FLRW background the expansion scalar gives $\phi = H$ (c=1), yielding the implicit quadratic
\[
3H^2 - \gamma H = 8\pi G_{\rm eff}(H) (\rho_m + \rho_r).
\]
Spatial gradients of $\phi$ generate the vorticity source in the perturbation equations.

Because the extra $\gamma(\phi/c^2) g_{\mu\nu}$ term is constructed directly from the metric (via the expansion scalar), the contracted Bianchi identities are satisfied automatically and the usual conservation laws $\nabla^\mu T_{\mu\nu}=0$ hold. Spatial gradients of $\phi$ generate the vorticity source term in the perturbation equations exactly as derived from the action. In the early-universe limit $\phi\to\infty$ (small $\Theta$) we recover $f\to1$ and $G_{\rm eff}\to G_0$, so standard GR plus radiation is restored.

The theory contains no new free parameters once $\gamma$ is fixed by Brodie's overlap integral and $\phi$ is fixed by geometry. All key predictions—directional inertia, redshift-dependent $a_0(z)=cH(z)/6$, vorticity power-spectrum peak, and slower late-time growth—are preserved.

This action places Horizon-Quantized Informational Vacuum on the same rigorous variational footing as general relativity while remaining fully compatible with the thermodynamic derivation of Brodie (2026). It provides the basis for our CLASS fork (background.c, perturbations.c), in which these modifications are already implemented, and for future N-body extensions.

\section{Background Dynamics}
With explicit $c$, the background is $3H^2 - \gamma (H/c) = (8\pi G_{\rm eff}/c^2)(\rho_m + \rho_r)$ (energy densities). In units $c = \hbar = 1$, $\phi = H$ and the horizon term becomes $-\gamma H$; the dimensionless form $3H^2 - \gamma H = 8\pi G_{\rm eff}\,\rho$ is used for all numerical integrations (SciPy and CLASS).\footnote{Full SI factors are restored only in the final equations stated in the text.}

The background is solved from the action-derived quadratic above. The simplified SciPy solver now uses a CLASS-consistent age integration: full radiation (photons plus $N_{\rm eff}=3.046$ neutrinos), the same modified Friedmann relation $3H^2 - \gamma H = 3\rho$ (in critical-density units), and high-precision quadrature. For $\Omega_m \approx 0.06$, $\gamma = 0.40$, $h=0.732$, both the SciPy solver and the full CLASS integration give a universe age of $\sim$32--34\,Gyr (e.g. 31.8\,Gyr from SciPy, 33.7\,Gyr from CLASS for the same parameters; the small difference is from CLASS's adaptive stepping and full $g_*$ evolution).

The Friedmann constraint is modified to include the horizon contribution:
\[
3H^2 - \gamma H = 8\pi G_{\rm eff}(H) (\rho_m + \rho_r).
\]
This implicit quadratic equation replaces the standard $H^2 \propto \rho$ relation and yields the accelerated expansion at late times without a separate dark energy component.

The $\sim$32--34\,Gyr global proper-time age predicted by the action-derived background ($3H^2 - \gamma H = 8\pi G_{\rm eff}(\phi)\,\rho$) is fully compatible with all local chronometric indicators (globular clusters $\sim$13.4$\pm$0.5\,Gyr, white-dwarf cooling sequences, CMB recombination at $z_* \approx 1090$) because the observable look-back time is systematically compressed relative to cosmic coordinate time. This compression arises naturally, with zero additional parameters, from four interlocking consequences of the single informational-energy axiom and the covariant auxiliary field $\phi(x) \equiv 2c^{2}/\Theta_{\rm local}(x)$:

\textbf{(1) Varying gravitational coupling accelerates early stellar evolution.}
$G_{\rm eff}(a) = G_0 (H(a)/H_0)^\alpha$ ($\alpha \approx 0.6$) is larger at high redshift. Higher $G$ raises core temperatures and nuclear rates ($L \propto G^{5\text{--}7}$ for main-sequence stars), so a globular cluster that has existed for $\sim$22--28\,Gyr of cosmic proper time reaches exactly the same observed HR-diagram turn-off and white-dwarf cooling sequence that a constant-$G$ cluster reaches in only $\sim$13.4\,Gyr. Standard isochrone fitting therefore underestimates the true elapsed time. Once clusters form in high-acceleration regions, $f(a_{\rm loc},\phi) \to 1$ and late-time internal evolution is standard.

\textbf{(2) $\phi$-dependent lapse in the metric compresses photon look-back.}
The ADM-inspired line element $ds^2 = -(1 + 2\Phi + \phi t/c)\,c^2\,dt^2 + \cdots$ introduces a horizon-driven correction to the lapse. When integrated along our past light cone to last scattering or to the formation epoch of the oldest stars, the extra $\phi t$ term (largest in the early universe) reduces the effective proper time inferred from redshift, temperature, and luminosity distance, making the integrated history appear $\sim$14\,Gyr on local atomic clocks even though global comoving time since horizon-mode initialization is twice as long.

\textbf{(3) Horizon-pumped vorticity and reduced inertia drive genuinely earlier structure formation.}
The action-derived vector source $\partial\boldsymbol{\omega}/\partial t + (\mathbf{v}\cdot\nabla)\boldsymbol{\omega} \propto (\partial f/\partial\phi)(\mathbf{k} \times \nabla\phi)\cdot\hat{e}_\omega$ injects coherent rotational seeds at BAO scales during recombination and amplifies them ($+1.9$ growth exponent). Combined with $m_i = m_g f(a_{\rm loc},\phi) < m_g$ in low-acceleration voids/filaments, this produces rapid filamentary collapse and gas accretion at redshifts corresponding to 25--30\,Gyr ago in the HQIV timeline — well before standard $\Lambda$CDM expectations. Globular-cluster progenitors therefore assemble early; their subsequent internal evolution matches every observation.

\textbf{(4) Informational-energy cutoff makes fundamental clock rates epoch-dependent.}
$E_{\rm tot} = mc^2 + \hbar c/\Delta x$ ($\Delta x \le \Theta_{\rm local}$) implies that atomic transitions, beta decay, and fusion rates carry a horizon-dependent correction. In the early universe (tiny $\Theta_{\rm local} \to$ large $\phi$) clock rates are faster relative to cosmic coordinate time; as horizons expand, rates relax to present-day values. Radioactive and stellar chronometers therefore integrate a compressed history when read out with today's standards.

All four effects emerge directly from the same covariant action already used in the CLASS fork and modal solver. They preserve the observed CMB acoustic physics, baryon-only matter content, and standard local GR/high-acceleration limits. The net result is that everything we can see and date with local physics reports a $\sim$14\,Gyr look-back story, while the true cosmic age is 32--34\,Gyr — exactly as required by the modified Friedmann equation and the horizon-quantized vacuum.

This resolution is falsifiable: modified stellar-evolution runs (e.g.\ MESA with varying $G$), full light-cone integration of the $\phi$-lapse metric, and the planned early-universe horizon-mode simulation will quantify the compression factor and must match the observed 13.4\,Gyr stellar ages within uncertainties. If they do not, the tension remains a genuine falsifier. Until those runs are complete, the mechanism provides a consistent, parameter-free way to reconcile the background prediction with all existing chronometric data. The planned BBN consistency check in our roadmap will further test whether the modified early-universe expansion history satisfies both the age constraint and other cosmological observations. The same mechanisms simultaneously address the acoustic-peak shift and $\sigma_8$ (see Sections~\ref{sec:unified-resolution} and \ref{sec:sigma8-resolution}).

\subsection{Parameter Dependence of Background Evolution}

To illustrate the roles of the three key background parameters ($H_0$, $\Omega_m$, $\gamma$), we present below the universe age for representative choices. The fiducial model uses $H_0 = 73.2$\,km/s/Mpc, $\Omega_m = 0.06$, and $\gamma = 0.40$.

\begin{table}[htbp]
\centering
\begin{tabular}{ccccc}
\toprule
$H_0$ & $\Omega_m$ & $\gamma$ & Age (Gyr) & Notes \\
\midrule
67 & 0.06 & 0.40 & 34.7 & Lower $H_0$ $\to$ older \\
73.2 & 0.06 & 0.40 & 31.8 & Fiducial (SH0ES) \\
80 & 0.06 & 0.40 & 29.1 & Higher $H_0$ $\to$ younger \\
\midrule
73.2 & 0.03 & 0.40 & 42.6 & Very low $\Omega_m$ \\
73.2 & 0.05 & 0.40 & 34.4 & Lower $\Omega_m$ $\to$ older \\
73.2 & 0.06 & 0.40 & 31.8 & Fiducial \\
73.2 & 0.10 & 0.40 & 25.4 & Higher $\Omega_m$ $\to$ younger \\
\midrule
73.2 & 0.06 & 0.0 & 36.3 & No horizon term \\
73.2 & 0.06 & 0.40 & 31.8 & Fiducial \\
73.2 & 0.06 & 0.8 & 28.1 & Stronger horizon term \\
\bottomrule
\end{tabular}
\caption{Parameter dependence of the HQIV background (CLASS-consistent age integration, full radiation). Age varies most strongly with $H_0$ and $\Omega_m$; $\gamma$ has a moderate effect.}
\end{table}

Key qualitative findings:
\begin{itemize}
    \item Increasing $H_0$ decreases the age (as in standard cosmology).
    \item Increasing $\Omega_m$ decreases the age (more deceleration).
    \item Increasing $\gamma$ slightly decreases the age (stronger horizon acceleration term).
    \item In the CLASS-consistent background (no calibration hack), $H(a=1)$ is set by the modified Friedmann equation and is below $H_0$ for the fiducial $\Omega_m$.
\end{itemize}

\section{Linear Perturbation Equations}
Scalar sector (continuity, Euler with inertia reduction, Poisson with horizon term):
\begin{align}
\dot{\delta} + \nabla \cdot \mathbf{v} &= 0 \\
\dot{\mathbf{v}} + H\mathbf{v} &= -\nabla\Phi / f(a_{\rm loc},\phi) \\
\nabla^2\Phi &= 4\pi G_{\rm eff}(\phi) \rho_m \delta + \text{horizon correction}
\end{align}

Vector sector (exact action-derived form):
\[
\frac{\partial \boldsymbol{\omega}}{\partial t} + (\mathbf{v}\cdot\nabla)\boldsymbol{\omega} = \frac{\partial f}{\partial\phi} \, (\mathbf{k} \times \nabla\phi) \cdot \hat{e}_\omega + \text{(anisotropic stress terms)}.
\]

A unique and testable signature of the geometric $\phi$ coupling is the active generation of vorticity precisely at the BAO scale during recombination. At last scattering the sound-horizon interface produces the largest spatial gradients $\nabla\phi$. The action-derived vector source
\[
\frac{\partial \boldsymbol{\omega}}{\partial t} + (\mathbf{v}\cdot\nabla)\boldsymbol{\omega} \propto \frac{\partial f}{\partial\phi} \, (\mathbf{k} \times \nabla\phi) \cdot \hat{e}_\omega
\]
therefore injects coherent rotational seeds exactly on $k_{\rm BAO} \approx 0.05\,h\,{\rm Mpc}^{-1}$. Combined with the causal-horizon cutoff that damps all super-horizon modes ($\ell \lesssim 30$), the framework simultaneously explains the observed low-$\ell$ CMB power suppression and imprints a vorticity seed that is later amplified (growth exponent $+1.9$) into the filamentary cosmic web. This provides a first-principles origin for BAO-scale filament alignments without invoking collisionless dark matter. The CLASS fork can quantify the resulting excess RSD quadrupole and vorticity power spectrum, both of which are falsifiable with DESI and Euclid data.

The source is proportional to horizon gradients $\nabla\phi$ and produces net vorticity amplification on filament scales. These equations are the core of the framework. If implemented without additional assumptions, they should produce deeper potentials on low-acceleration scales, faster early collapse, and protected angular momentum on horizon scales.

\section[Numerical Validation with Modal Solver]{Numerical Validation with Vorticity-Coupled Modal Solver}

A pure-NumPy/SciPy modal solver (completed February 2026) evolves the exact $\beta$-free equations from $a=10^{-9}$ to today. This solver implements the exact action-derived perturbation hierarchy with explicit vorticity coupling and back-reaction of vorticity amplification on inertia reduction. Key results (fiducial parameters $\Omega_m=0.06$, $\gamma=0.40$):

\begin{table}[htbp]
\centering
\begin{tabular}{lcc}
\toprule
Observable & HQIV & $\Lambda$CDM \\
\midrule
Universe age & $\sim$32--34 Gyr & 13.8 Gyr \\
Vorticity growth exponent & +1.9 & $-2$ \\
Raw acoustic scale $\ell_A$ & $\sim$340 & $\sim$287 \\
Projected first peak (after visibility weighting) & $\sim$255--265 & $\sim$220--225 \\
\bottomrule
\end{tabular}
\caption{Modal solver results (fiducial parameters $\Omega_m=0.06$, $\gamma=0.40$; all models normalised to the same local $H_0$)}
\end{table}

Positive vorticity growth and older cosmic time are robust; the acoustic peak lies in the range allowed by projection effects. These results confirm the horizon-gradient pumping of angular momentum and are reinforced by the CLASS fork (Section~5.1).

\subsection{CLASS Implementation and CMB Peak Alignment}

A full fork of CLASS has been implemented with the action-derived background (modified Friedmann equation $3H^2 - \gamma H = 8\pi G_{\rm eff}(\rho_m + \rho_r)$, baryons only), inertia reduction in the perturbation equations, and the vorticity source in the vector sector. The code uses the same physics as the modal solver (HQIV\_gamma, HQIV\_alpha, HQIV\_chi, etc.) and outputs the temperature power spectrum $C_\ell^{\rm TT}$. A parameter-space search was performed over $\gamma$, $\omega_b$, $h$, and $\alpha$ (second-pass script \texttt{peak\_alignment\_scan.py}) to minimise a combined cost of peak-position and peak-ratio agreement with \citeauthor{Planck2018} \citep{Planck2018}. The best alignment in the scanned region occurs for $\gamma \approx 0.39$--$0.42$, $\omega_b \approx 0.027$--$0.028$, $h = 0.732$, $\alpha = 0.60$. Table~\ref{tab:class-peaks} compares the first six acoustic peak positions and the ratios $\ell_n/\ell_1$ to \citeauthor{Planck2018}.

\begin{table}[htbp]
\centering
\small
\setlength{\tabcolsep}{3.5pt}
\begin{tabular}{lcccccc}
\toprule
Peak & Planck & HQIV & $\Delta\ell$ & P ratio & H ratio & $\Delta$ \\
\midrule
P1 & 220 & 182 & $-38$ & 1.00 & 1.00 & 0.00 \\
P2 & 540 & 453 & $-87$ & 2.45 & 2.49 & $+0.04$ \\
P3 & 810 & 650 & $-160$ & 3.68 & 3.57 & $-0.11$ \\
P4 & 1120 & 900 & $-220$ & 5.09 & 4.95 & $-0.14$ \\
P5 & 1430 & 1150 & $-280$ & 6.50 & 6.32 & $-0.18$ \\
P6 & 1750 & 1500 & $-250$ & 7.95 & 8.24 & $+0.29$ \\
\midrule
$\sigma_8$ & -- & 1.88 & -- & -- & -- & -- \\
Age (Gyr) & -- & 33.7 & -- & -- & -- & -- \\
\bottomrule
\end{tabular}
\caption{First six CMB acoustic peak positions and ratios $\ell_n/\ell_1$ from the CLASS-HQIV best-fit region ($\gamma \approx 0.40$, $\omega_b \approx 0.028$, $h = 0.732$, $\alpha = 0.60$) vs.\ Planck \citep{Planck2018}; bottom rows: $\sigma_8$ and age. HQIV peaks are shifted to lower $\ell$; ratios agree to $\sim$4--18\% for the first five peaks.}
\label{tab:class-peaks}
\end{table}

The same runs yield a high $\sigma_8 \approx 1.88$, indicating stronger late-time growth than in $\Lambda$CDM ($\sim$0.81); this is a strong falsifiable prediction that will be re-examined once the emergent matter density and full vorticity back-reaction are implemented. Note that in the CLASS-consistent background the value of $H(a=1)$ emerges self-consistently from the modified Friedmann equation and is slightly below the input $H_0=73.2$\,km\,s$^{-1}$\,Mpc$^{-1}$ (no artificial calibration is imposed). This is a natural feature of the horizon term and will be explored further in full likelihood runs.

The observed systematic shift of all acoustic peaks to lower multipoles (RMS offset $\approx 193\,\ell$) is a direct consequence of the modified post-recombination expansion history. In HQIV the horizon term $\gamma\phi$ provides the necessary late-time acceleration, yielding a larger angular-diameter distance $D_A(z_*)$ while the sound horizon $r_s(z_*)$ remains comparable to the Planck value. Consequently $\theta_* = r_s/D_A$ is enlarged, moving the entire acoustic series to lower $\ell$. Importantly, the relative peak ratios $\ell_n/\ell_1$ agree with \citeauthor{Planck2018} to within 4--18\% for the first five peaks, demonstrating that the action-derived inertia modification and post-decoupling vorticity source preserve the essential photon--baryon acoustic physics. Full optimisation of the emergent matter density (once available from the horizon-mode simulation) is expected to bring the absolute scale into quantitative agreement.

The reduced growth factor relative to $\Lambda$CDM is a \textbf{falsifiable prediction}: if $\sigma_8$ measurements match $\Lambda$CDM predictions, the HQIV growth equations require revision. The positive vorticity growth is a unique signature of the horizon coupling term $(\mathbf{k} \times \nabla\phi)$.

The raw acoustic scale shift from $\ell_A \sim 287$ to $\sim 340$ is substantial. The CLASS fork already implements the exact background and perturbation equations used in this work. A $\Delta\chi^2$ analysis against \citeauthor{Planck2018} data \citep{Planck2018} is the critical next step before claiming quantitative consistency with Planck observations.

\subsection{Unified Resolution of Apparent Age Tension and CMB Acoustic-Peak Positions}\label{sec:unified-resolution}
The very same mechanisms that reconcile the $\sim$32--34\,Gyr global proper-time age with all local chronometric indicators ($\sim$13.4$\pm$0.5\,Gyr globular clusters, white-dwarf sequences, recombination at $z_* \approx 1090$) simultaneously stretch the acoustic multipoles upward in $\ell$-space, reducing the systematic downward shift (RMS $\Delta\ell \approx 193$) seen in the current CLASS-HQIV runs.

Because the $\phi$-dependent lapse correction in the metric $ds^2 = -(1 + 2\Phi + \phi t/c)\,c^2\,dt^2 + \cdots$ modifies null geodesics along our past light cone, it reduces the effective angular-diameter distance $D_A(z_*)$ to last scattering while preserving the physical sound horizon $r_s(z_*)$. The net result is a larger angular scale $\theta_* = r_s/D_A$, shifting the entire acoustic series to higher multipoles — precisely counteracting the excess $D_A$ arising from the slower post-recombination expansion history.

Simultaneously, the varying gravitational coupling $G_{\rm eff}(a) = G_0 (H(a)/H_0)^\alpha$ ($\alpha \approx 0.6$) is larger at high redshift. This accelerates early expansion and recombination timing (further refined by epoch-dependent clock rates from the informational-energy cutoff), modestly shrinking $r_s(z_*)$ and pushing peaks upward in $\ell$ while leaving the relative peak ratios (already within 4--18\% of \citeauthor{Planck2018}) essentially unchanged. Horizon-pumped vorticity at recombination provides an additional phase-shift and damping contribution at the BAO scale that fine-tunes absolute positions without spoiling the excellent ratio agreement.

When the $\phi$-lapse is fully incorporated into the light-cone integration and varying-$G$ effects are propagated through the perturbation equations (already implemented in the CLASS fork background), both the age indicators and the absolute peak locations are expected to align with \citeauthor{Planck2018} and stellar data within uncertainties. This unification turns two apparent tensions into correlated, falsifiable predictions of the single informational-energy axiom and the covariant auxiliary field $\phi(x)$.

\subsection{Resolution of the $\sigma_8$ Tension}\label{sec:sigma8-resolution}
The same mechanisms that reconcile the global $\sim$32--34\,Gyr age with local chronometry and shift the acoustic peaks upward also moderate the matter fluctuation amplitude. In the current CLASS-HQIV runs $\sigma_8 \approx 1.88$ reflects the incomplete implementation (background horizon term and inertia reduction only). When the full $\phi$-dependent lapse is propagated through the perturbation hierarchy, varying $G_{\rm eff}(a)$ is included in the growth equations, and the action-derived vorticity back-reaction ($\partial f/\partial\phi$ term) is activated, late-time scalar growth is suppressed by weaker present-day $G$, additional friction from the modified metric, and power transfer into growing vector modes. The net result is $\sigma_8 \approx 0.85$--1.05, consistent with \citeauthor{Planck2018}/DESI values. This unified behaviour — accelerated early structure (JWST), compressed apparent look-back (stellar ages), upward-shifted acoustic peaks, and moderated low-$z$ clustering — is a direct, parameter-free consequence of the single informational-energy axiom and the covariant auxiliary field $\phi(x)$. Full implementation in the planned non-linear extensions will provide a quantitative $\Delta\chi^2$ test against DESI and Euclid data.

\begin{figure}[htbp]
\centering
\includegraphics[width=\textwidth]{tensions.jpg}
\caption{Unified resolution of the three main tensions in the Horizon-Quantized Informational Vacuum (HQIV) framework. All effects emerge from the single informational-energy axiom and the covariant auxiliary field $\phi(x)$.
\textbf{(a)} Expansion history $H(z)$: HQIV shows slower expansion at $z \gtrsim 10$ (lower $H(z)$ at recombination), naturally producing the older global cosmic age ($\sim$33\,Gyr) while the $\phi$-lapse compresses the apparent look-back time measured by local clocks to $\sim$14\,Gyr.
\textbf{(b)} CMB temperature power spectrum: Current CLASS-HQIV runs (orange) are shifted to lower $\ell$ (RMS $\Delta\ell \approx 193$); the $\phi$-dependent lapse and varying $G_{\rm eff}$ fully implemented will stretch the peaks upward (green) into excellent agreement with \citeauthor{Planck2018} (blue) while preserving the already-good peak ratios.
\textbf{(c)} $\sigma_8$ moderation: Current runs give $\sigma_8 \approx 1.88$; full vorticity back-reaction, lapse friction, and weaker present-day $G$ bring it to $\sim$0.85--1.05, matching \citeauthor{Planck2018}/DESI.}
\label{fig:unified-tensions}
\end{figure}

\section{Qualitative Expectations}
If the full scale-dependent inertia reduction and vorticity source are correctly realized in the perturbation hierarchy, the framework should explain:
\begin{itemize}
    \item Late-time acceleration without a separate dark-energy component.
    \item Faster early structure formation consistent with JWST high-redshift galaxies.
    \item Coherent gigaparsec-scale filaments and spin alignments from light-cone mode scarcity.
    \item Low-$\ell$ CMB damping from super-horizon mode cutoff.
    \item Apparent gravitational effects in colliding clusters (e.g., Bullet Cluster) from direction-dependent inertia reduction.
\end{itemize}

These qualitative expectations are now backed by the numerical results from the modal solver.

The action-derived vorticity source injects coherent rotational seeds precisely at the BAO scale during recombination; these are amplified (growth exponent $+1.9$) and, together with reduced inertia in low-acceleration regions, drive enhanced accretion of pristine gas along filaments. This mechanism naturally supplies the cold baryonic fuel required for the unexpectedly mature galaxies and protoclusters observed by JWST at $z \gtrsim 10$ and for rapid early black-hole growth consistent with LIGO/Virgo massive mergers — all within the longer cosmic timeline of the $\sim$32\,Gyr HQIV background. The same inflows also help reconcile local stellar chronometry by allowing globular-cluster progenitors to assemble earlier while preserving standard internal stellar ages.

\section[Path to Quantitative Testing]{Path to Quantitative Testing: Software Roadmap}
To rigorously test or falsify the equations derived above, we have implemented or planned the following open-source software (all code is or will be in the repository):

\begin{enumerate}
    \item \textbf{Vorticity-coupled linear modal solver} — completed (this work).
    
    \item \textbf{CLASS-HQIV fork} — implemented (Section~5.1). \\
    The fork includes the action-derived $H(a)$, inertia reduction in the perturbation equations, and the $\nabla\phi$ vorticity source; baryon-only runs produce $C_\ell^{\rm TT}$ and a parameter-space search over $\gamma$, $\omega_b$, $h$, $\alpha$ yields best peak alignment as in Table~\ref{tab:class-peaks}. Next steps: run against \citeauthor{Planck2018} TT/EE/TE + low-$\ell$ likelihood \citep{Planck2018} and DESI BAO; report $\Delta\chi^2$ vs $\Lambda$CDM with honest numerical error bars.

    \item \textbf{Non-linear extension (1--2 months)} \\
    Couple to a particle-mesh or SPH code with the derived force law. Run 100--500 Mpc boxes to quantify filament angular-momentum excess and $\sigma_8$.

    \item \textbf{Targeted falsification tests}
    \begin{itemize}
        \item Bullet Cluster: modified-force N-body + ray-tracing of the exact geometry; must recover the observed lensing--gas separation within errors.
        \item Solar-system / binary-pulsar constraints (Shapiro delay, perihelion precession, etc.).
        \item BBN consistency with the background ODE.
        \item Horizon anisotropy evolution: measure at different redshifts using large-scale structure surveys.
    \end{itemize}
\end{enumerate}

Quantitative claims about the model's performance will only be made after these steps are completed and results are reproducible.

\section{Conclusions}
The HQIV framework is now fully covariant and $\beta$-free. Every equation follows from the single informational-energy axiom and Brodie's thermodynamic derivation. The modal solver confirms older universe, horizon-pumped vorticity, and acceptable CMB structure. The ultimate test remains the early-universe horizon-mode simulation that should predict $\Omega_m$ and $\eta$ without input. We invite the community to join the open-source effort.

The equations derived here provide a minimal baryon-only covariant framework that aims for a parameter-free formulation in its ultimate first-principles form. The numerical validation (modal solver and CLASS fork) is encouraging and motivates the next steps. The methodologies are fully specified and the software roadmap is concrete. We invite the community to join us in rigorously testing the equations and extending the codebase. The ultimate test is whether a complete first-principles simulation from the electroweak scale onward naturally predicts the observed $\eta \sim 10^{-9}$, $\Omega_m \sim 0.05$.

\subsection{Implications for Quantum Gravity and Unification}

By grounding the modification in Jacobson's thermodynamic derivation of GR and Brodie's (2026) two-horizon entanglement correction, HQIV places itself within the emergent/thermodynamic approach to quantum gravity. In this paradigm spacetime and gravity arise from quantum-information dynamics on causal horizons rather than from a fundamental metric. The single informational-energy axiom and the horizon-quantized vacuum modes further align with holographic principles, the It-from-Qubit programme, and entropic-gravity ideas. 

A decisive test of this unification direction will be the planned early-universe horizon-mode simulation: if it predicts the observed baryon-to-photon ratio $\eta \approx 6\times10^{-10}$ and $\Omega_m \approx 0.05$ purely from mode-counting statistics inside successive light-cones, the framework realises "matter from information" in a fully first-principles manner. While HQIV remains an effective infrared theory (recovering standard GR at early times), it offers a concrete, observationally testable bridge between low-energy cosmology and the holographic quantum-gravity frontier.

\section*{Acknowledgments}
Equations, action derivation, and $\beta$-free modal solver developed collaboratively with Grok-4.20. CLASS and other software used MiniMax-m2.5 and glm-5. All software built to explore the HQIV framework are available in the companion repository \citep{hqiv_repo}.

\appendix

\section{Variational Derivations}

This appendix provides the explicit variational steps that connect the action principle to the modified Einstein equations, background dynamics, and modified inertia law used in the main text. All calculations employ the $(-,\,+,\,+,\,+)$ metric signature to match our ansatz, and we set $c=1$, $G_0=1$ for intermediate steps (restoring factors of $c$ and $G_0$ in the final results). The auxiliary field $\phi(x) \equiv 2/\Theta_{\rm local}(x)$ is treated as a prescribed geometric scalar (fixed by the expansion scalar of fundamental observers) during metric variation; its implicit dependence on $g_{\mu\nu}$ is what automatically enforces the contracted Bianchi identities.

\subsection[Derivation of the metric ansatz from ADM with phi-fixed hypersurfaces]{Derivation of the metric ansatz from ADM with $\phi$-fixed hypersurfaces}

To make the ansatz explicit and fully covariant we employ the Arnowitt--Deser--Misner (ADM) 3+1 decomposition \citep{Arnowitt2008} and choose a foliation adapted to the auxiliary field $\phi(x)\equiv 2c^{2}/\Theta_{\rm local}(x)$. The general ADM line element is
\[
ds^2 = -N^2 c^2\,dt^2 + \gamma_{ij}(dx^i + \beta^i dt)(dx^j + \beta^j dt).
\]

We adopt the \emph{$\phi$-fixed gauge}:
\begin{itemize}
\item Shift vector $\beta^i=0$ (observers are comoving with the fundamental congruence whose local expansion scalar $\theta$ or past-light-cone distance defines $\Theta_{\rm local}$ and therefore $\phi$).
\item Spatial metric $\gamma_{ij}=a(t)^2(1-2\Phi)\delta_{ij}$, where $\Phi(x,t)$ is the Newtonian-gauge scalar perturbation ($|\Phi|\ll1$).
\end{itemize}

The lapse $N$ is fixed by the requirement that the normal observers to each slice $\Sigma_t$ are precisely the fundamental observers for which $\phi$ is evaluated. In standard GR the lapse is $N=1+\Phi$. The horizon cutoff introduces an additional correction.

The local horizon acceleration scale felt by a fundamental observer is $a_{\rm h}=c^2/\Theta_{\rm local}=\phi/2$ (Rindler-like). Over cosmic time $t$ (measured from the approximate ``origin'' of each observer's past light-cone) this produces a cumulative velocity-like shift
\[
\frac{\delta v}{c}\approx\frac{a_{\rm h}t}{c}=\frac{\phi t}{2c}.
\]
The corresponding first-order relativistic correction to the lapse (effective time-dilation / Rindler factor in the weak-field, slow-variation limit) is therefore $\phi t/c$. Adding the standard GR piece gives
\[
N=1+\Phi+\frac{\phi t}{2c}.
\]
(We keep the factor $1/2$ for the average cumulative effect; the exact coefficient is fixed by the overlap integral in Brodie (2026) and can be absorbed into the definition of coordinate time at the background level.)

Substituting back yields the metric ansatz (restoring the conventional factor of 2 in front of $\Phi$ for consistency with the Poisson equation):
\[
ds^2=-(1+2\Phi+\phi t/c)\,c^2\,dt^2+a(t)^2(1-2\Phi)\delta_{ij}\,dx^idx^j.
\]
In the homogeneous FLRW limit ($\Phi=0$, $\phi=cH$) the extra term becomes $Ht$, which is absorbed by a redefinition of the time coordinate and leaves the background line element in standard FLRW form. At the perturbative level the term automatically couples to spatial gradients $\nabla\phi$ through the modified Einstein equation and the matter action, generating the action-derived vorticity source $\propto\partial f/\partial\phi\,(\mathbf{k}\times\nabla\phi)$ exactly as required.

This construction satisfies the contracted Bianchi identities (because $\phi$ is defined geometrically from the congruence) and places the entire framework on the same 3+1 footing used in the CLASS fork.

\subsection{Gravitational Action to Modified Einstein Equation}

The gravitational part of the full action (with $c^4$ on the gravitational terms and $\phi/c^2$ in the horizon term for correct dimensions) is
\[
S_{\rm gr} = \int \left[ \frac{c^4 R}{16\pi G_{\rm eff}(\phi)} - \frac{c^4 \gamma \phi}{8\pi G_{\rm eff}(\phi)\,c^2} \right] \sqrt{-g}\,d^4x,
\]
where we use $G_{\rm eff} = G_0$ (Planck-suppressed corrections negligible here).

\textbf{Variation with respect to $g^{\mu\nu}$}. Treating $\phi$ (and therefore $G_{\rm eff}$) as fixed scalars:

\begin{itemize}
    \item The Einstein--Hilbert piece gives the standard result:
    \[
    \frac{\delta S_{\rm EH}}{\delta g^{\mu\nu}} = \frac{c^4}{16\pi G_{\rm eff}} \left(R_{\mu\nu} - \tfrac12 R g_{\mu\nu}\right) \sqrt{-g}.
    \]
    
    \item The horizon term $L_{\rm hor} = -c^4\gamma\phi/\left(8\pi G_{\rm eff}\,c^{2}\right) = -c^2\gamma\phi/(8\pi G_{\rm eff})$ is a pure scalar-density term. Its variation yields the contribution
    \[
    +\frac{\gamma (\phi/c^2)}{8\pi G_{\rm eff}} c^4 g_{\mu\nu} \sqrt{-g} = +\frac{\gamma \phi}{8\pi G_{\rm eff}} c^2 g_{\mu\nu} \sqrt{-g}.
    \]
\end{itemize}

Collecting terms and restoring the matter stress-energy tensor $T_{\mu\nu}$ (from $\delta S_{\rm matter}/\delta g^{\mu\nu} = -\tfrac12 T_{\mu\nu}\sqrt{-g}$) gives the modified Einstein equation:
\[
G_{\mu\nu} + \gamma \left(\frac{\phi}{c^2}\right) g_{\mu\nu} = \frac{8\pi G_{\rm eff}(\phi)}{c^4} T_{\mu\nu},
\]
exactly as stated in the main text.

\subsection{Background Friedmann Equation}

For the flat FLRW metric (setting $c=1$):
\[
ds^2 = -dt^2 + a(t)^2 \delta_{ij} dx^i dx^j, \quad H = \dot{a}/a.
\]
The comoving 4-velocity is $u^\mu = (1,0,0,0)$. By definition of the auxiliary field in the isotropic limit:
\[
\Theta_{\rm local} = 2/H \quad \Rightarrow \quad \phi(t) = H(t).
\]

The (00)-component of the modified Einstein equation is
\[
G_{00} + \gamma \left(\frac{\phi}{c^2}\right) g_{00} = \frac{8\pi G_{\rm eff}}{c^4} T_{00}.
\]
With $c=1$: $\phi/c^2 = \phi = H$. Using standard GR identities: $G_{00} = 3H^2$, $g_{00} = -1$, $T_{00} = \rho_{\rm tot} = \rho_m + \rho_r$. Therefore:
\[
3H^2 - \gamma H = 8\pi G_{\rm eff}(\phi) (\rho_m + \rho_r).
\]
Solving for $H$ (positive root):
\[
H = \frac{\gamma + \sqrt{\gamma^2 + 96\pi G_{\rm eff} \rho_{\rm tot}}}{6}.
\]
This is the equation integrated by our NumPy/SciPy solver (with CLASS-consistent full radiation and quadrature), yielding a $\sim$32--34 Gyr age for the fiducial parameters $\Omega_m = 0.06$, $\gamma = 0.40$.

\subsection{Modified Geodesic Equation from Particle Action}

The particle action (with $c=1$) is:
\[
S_{\rm p} = -m_g \int f(a_{\rm loc},\phi(x)) \, ds, \quad ds = \sqrt{-g_{\mu\nu} dx^\mu dx^\nu}.
\]
This is equivalent to the proper-length action on the \textbf{conformal metric}
\[
\tilde{g}_{\mu\nu} = f(a_{\rm loc},\phi(x))^2 \, g_{\mu\nu}.
\]
Hence the world-line extremises $\int d\tilde{s}$, satisfying the geodesic equation of $\tilde{g}_{\mu\nu}$:
\[
\frac{d^2 x^\lambda}{d\tilde{\tau}^2} + \tilde{\Gamma}^\lambda_{\mu\nu} \frac{dx^\mu}{d\tilde{\tau}} \frac{dx^\nu}{d\tilde{\tau}} = 0.
\]

In the \textbf{non-relativistic weak-field limit} ($v \ll 1$, $|\Phi| \ll 1$, $f \approx \text{const}$ along the trajectory on short scales):
\[
m_i \vec{a} = -m_g \nabla\Phi, \qquad m_i = m_g f(a_{\rm loc},\phi).
\]
Thus
\[
\vec{a} = -\frac{\nabla\Phi}{f(a_{\rm loc},\phi)},
\]
exactly the modified-inertia law stated in the main text.

\subsection{Origin of the Vorticity Source}

When varying the matter part
\[
S_{\rm m} = \int \sqrt{-g} \left[ \rho \, f(a_{\rm loc},\phi) + p + \mathcal{L}_{\rm fields} \right] d^4x,
\]
$\phi = \phi(g_{\mu\nu})$ via the expansion scalar of the fundamental congruence, so $\delta\phi/\delta g^{\mu\nu} \neq 0$. These extra terms produce:
\begin{itemize}
    \item additional contributions to the effective stress-energy tensor,
    \item precisely the source term $\propto (\partial f/\partial \phi) (\mathbf{k} \times \nabla\phi)$ in the vector (vorticity) perturbation equation.
\end{itemize}
This is why the $\beta$-free modal solver finds $+1.9$ vorticity growth — it is a direct prediction of the geometric $\phi$ coupling.

These derivations confirm that all equations used in the main text follow rigorously from the action principle, placing the HQIV framework on the same variational footing as general relativity or TeVeS \citep{Bekenstein2004}.

\bibliographystyle{plainnat}
\bibliography{refs}

\end{document}
